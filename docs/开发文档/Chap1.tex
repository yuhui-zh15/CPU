\chapter{开发周期}

我们将开发周期分成5个Sprint,具体展开每个开发周期应该做什么,通过这样详细的拆分,希望读者可以能尽快走出迷茫期,对整个项目有一个大概的认知。

\section{Sprint 1}

Sprint 1的持续时间为Week 2 - Week 4,本阶段我们刚刚接触项目,并加上十一国庆节,在这个阶段,我们做了:

    \begin{enumerate}
        \itembf{阅读文献}: 打印阅读《自己动手写CPU》和学长文档。《自己动手写CPU》是一本极其重要的书,作者清晰的思路带领读者一步一步的从零开始搭建出一个流水线框架,因此建议每人一本,快速阅读。
        \itembf{配置环境}: 学习软工平台和Gitlab,安装Vivado,配置Ubuntu云服务器。由于我们三个人的电脑均不是Ubuntu,故我们选择购买了一台最低配的云服务器作为我们Ubuntu的开发环境,从而实现安装mips编译器,编译功能测例、ucore等等,云服务器给我们提供了完全相同资源共享的环境,且上传下载简单。
        \itembf{明确需求}: 事实上,在这个阶段,尽管我们读了《自己动手写CPU》和学长文档,但是因为知识面远远不够,我们对项目的认知还是较为模糊,无法进行明确分工。
    \end{enumerate}

    \image[5in]{cpu}{神书《自己动手写CPU》}

从Milestone也可以看出,这段时期我们并没有明显的开发痕迹,主要是阅读资料。

\section{Sprint 2}

Sprint 2的持续时间为Week 5 - Week 7。在这个阶段,我们做了:

    \begin{enumerate}
        \itembf{搭建流水线框架}:实现MIPS标准五级流水线,只支持ori指令。
        \itembf{增加各类指令}:增加逻辑运算指令、算数运算指令、分支跳转指令、访存指令等ucore运行所需的大部分指令。
        \itembf{设计数据通路}:解决流水线竞争与冒险,数据前推,增加控制器实现流水线暂停功能。
        \itembf{单一指令测试}:搭建SOPC,对实现的每条指令进行简单的测试,检查实现是否正确。
    \end{enumerate}

这个阶段是后面所有阶段的基础,推荐的方法是快速阅读与实现《自己动手写CPU》相关内容,本阶段可以分工,每个成员实现一章的指令,但是每个成员都要读相关章节,这样才能保证所有人能对系统有一个大概的认知。

\section{Sprint 3}

Sprint 3的持续时间为Week 8 - Week 10。在这个阶段,我们做了:
    
    \begin{enumerate}
        \itembf{异常处理}:实现CP0协处理器与异常处理,对其进行相应的的测试。
        \itembf{内存管理}:阅读TLB、MMU相关文献,理解TLB、MMU的基本原理,实现TLB、MMU从而进行内存管理。
        \itembf{外设连接}:了解各种外设(Flash、RAM、串口)的使用方法,写代码对其进行简单测试
        \itembf{仿真操统}:在搭建的云服务器上,编译ucore,并使用qemu进行仿真,了解我们的终极目标是什么。
    \end{enumerate}

这个阶段已经算是进阶了,已经对CPU基本的框架有了相应的了解,本阶段可以分工,异常处理由一个人完成,推荐的方法是快速阅读与实现《自己动手写CPU》相关内容,之后《自己动手写CPU》这本书便完成了它的使命,后期已经没有可以参考直接使用的东西了,最多可以参考该书了解一些相关概念!内存管理推荐一个人完成,上网查阅理解TLB、MMU的基本概念,或者向学长老师要一份往年的计原很靠后的PPT,从而实现内存管理模块。外设连接推荐一个人完成,对提供的顶层代码写一些验证性代码,理解外设的使用方法,掌握外设的时序关系,能阅读相应的英文外设文档。分工后组员一定要进行相互讲解,review对方的代码,进一步加深对系统的理解。

\section{Sprint 4}

Sprint 4的持续时间为Week 11 - Week 13。在这个阶段,我们做了:
    
    \begin{enumerate}
        \itembf{调试功能测例}:功能测例内含九十余条指令的测试代码,编译后烧录至RAM中便可以测试CPU相关指令是否实现正确,功能测例对指令实现要求高,对外设实现要求低。我们先仿真通过了所有功能测例,又使用硬件通过了所有功能测例。
        \itembf{调试监控程序}:监控程序对指令实现要求低,对外设实现要求高,调试需要实现ROM、Flash和串口,实现boot过程。
    \end{enumerate}

进入这个阶段已经没有什么文档可以参考了,因为每个人的问题都不一样,推荐的方法是根据调试文档中的调试技巧,定位问题所在,查阅学长文档和网上的资料,三人共同讨论分析问题、解决问题。

\section{Sprint 5}

Sprint 4的持续时间为Week 14 - Week 16。在这个阶段,我们做了:
    
    \begin{enumerate}
        \itembf{TLB功能测例}:依据功能测例的基本框架,对TLBWR和TLBWI两条指令撰写功能测例。
        \itembf{调试ucore}:ucore对指令实现要求高,对外设实现要求高,但由于功能测例和监控程序都已进行了充分的测试,故本阶段调试较为顺利,仅耗时2天。
        \itembf{增加外设}:增加显存模块,增加VGA输出模块,实现图像输出功能。
        \itembf{文档撰写}:总结经验教训,总结开发心得。
    \end{enumerate}

如果前面每个组员都完成了自己的任务,对系统有一个清晰的理解,每项功能都经历了严格的测试,积累了大量的经验,本阶段将不再那么痛苦,但是如果之前阶段得过且过,本阶段可能会遇到无数的问题,因此之前的阶段一定要认真开发,不要划水!





