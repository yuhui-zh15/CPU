\chapter{32位监控程序}

\section{简介概述}

32位监控程序\footnote{文档地址:https://wyl8899.gitbooks.io/road-to-neptunus/content/;\\项目地址:https://git.net9.org/wyl8899/Neptunus}相当于一个微型操作系统,
其内核使用MIPS 32汇编语言编写(位于monitor/kern/目录),客户端终端应用程序使用Python语言编写(term/term.py文件)。

监控程序主要用于小范围的系统测试,将指令集中的23条指令综合使用,以确保其作为一个整体的正确性。

相比功能测例,监控程序增加了对BootLoader、中断以及串口的需求,因此对外设的测试覆盖面大大扩展。此外,它还支持TLB的测试,但需要增加2条TLB指令(TLBP、TLBR)。
我们因此没有采用监控程序进行TLB的测试。

监控程序的主要流程如下:

\begin{enumerate}
    \item {\bf Boot阶段}:硬件初始化时置PC值为VA 0xBFC00000,读取ROM中存储的BootLoader,将Flash中存储的监控程序拷贝至RAM。拷贝结束后跳转至启动入口VA 0x80000000。
    \item {\bf 系统初始化}:系统初始化阶段暂停中断处理,而后对CP0的Status、Cause、Ebase寄存器进行初始化,最后进入正常中断模式。
    \item {\bf 启动完毕}:启动shell主线程,向串口打印启动信息。

    用户在终端可以执行以下命令来验证监控程序正确运行(其中的参数均为32位,使用小端序):
    \begin{itemize}
        \item {\bf R}:\texttt{R}

        按照\$1至\$30的顺序返回用户空间寄存器值。

        \item {\bf D}:\texttt{D <addr> <num>}

        按照小端序返回指定地址连续num字节。

        \item {\bf A}:\texttt{A <addr> <num> <content>}

        在指定地址写入content。约定content有num字节,并且num为4的倍数。

        \item {\bf G}:\texttt{G <addr>}

        执行指定地址的用户程序。ra传递正常退出地址。

        \item {\bf T}:\texttt{T <num>}

        查看index=num的TLB项,依次回传ENTRYHI, ENTRYLO0, ENTRYLO1共12字节。

    \end{itemize}
\end{enumerate}

\section{测试范畴}

\subsection{测试覆盖面}

\image[5in]{monitor}{32位监控程序测试范畴(绿-可以测试;灰-不能测试)}

32位监控程序主要测试了CPU的以下部件是否实现正确:

\begin{enumerate}
    \item {\bf 五级流水线}:包括IF、ID、EX、MEM、WB共5个模块。
    \item {\bf 寄存器}:包括寄存器堆(32个通用寄存器)、HI/LO寄存器共2个模块。
    \item {\bf CP0}:包括CP0共1个模块。
    \item {\bf Control}:包括Control(流水线控制器)共1个模块。
    \item {\bf ROM}:包括ROM(存储BootLoader)共1个外部设备。
    \item {\bf RAM}:包括RAM(用作内存)共1个外部设备。
    \item {\bf Flash}:包括Flash(存储监控程序)共1个外部设备。
    \item {\bf 串口}:包括串口(用于与用户终端交互)共1个外部设备。
\end{enumerate}

32位监控程序不能全面测试的CPU部件包括:

\begin{enumerate}
    \item {\bf MMU}:我们没有打开监控程序的TLB选项,故不能全面测试MMU。
\end{enumerate}

\subsection{测试要点}

除去功能测例已经较为完整地测试过了的指令集合之外,32位监控程序的测试要点集中于访存(包括对外部设备的时序调度)与异常处理方面:

\begin{enumerate}
    \item {\bf ROM与BootLoader}:能正确地读取ROM,并运行BootLoader完成引导过程。
    \item {\bf RAM}:能正确实现内存的读写操作。
    \item {\bf Flash}:能正确地读取Flash中存储的监控程序,以完成BootLoader阶段的拷贝任务。
    \item {\bf 串口}:能正常地读写串口,以实现用户终端的标准输入、输出。
    \item {\bf 异常处理}:能正确处理硬件中断(包括时钟中断、串口中断)、Syscall异常、计算溢出异常。
\end{enumerate}

\section{测试方式}

32位监控程序的测试分为如下步骤:

\begin{enumerate}
    \item {\bf 系统初始化}

    \begin{enumerate}
        \item 将BootLoader编译后写入ROM对应的Verilog文件中。
        \item 将监控程序编译后烧写入Flash的地址0x0处。
        \item 将CPU的.bit设计文件烧写入开发板FPGA中。
        \item 点击开发板上Reset开关启动系统。
    \end{enumerate}

    \item {\bf 启动客户端(监控程序终端)}

    \begin{enumerate}
        \item 将开发板的USB接口连接至PC端,以建立串口通信。
        \item 在PC端(Ubuntu 16.04 x64系统)监控程序的term/目录下运行\texttt{python term.py <pipe\_path>}即可启动用户终端。
        \item 可以输入R、D、A、G、T共5种指令以验证监控程序已正确运行。
    \end{enumerate}

\end{enumerate}

\section{测试结果}

<TODO>:截图

经测试,监控程序可以正常启动,但是运行结果与预期不符。经检测,这是因为其软件实现存在bug。

\section{问题与解决}

<TODO>:建议简单描述一下这里出现的bug

例如下面这些:

\begin{enumerate}
    \item {\bf Flash地址总线的一个bug}:\url{http://47.94.142.165:8088/gitlab/PRJ11_NonExist/CPU/commit/920973a3a1a28f027a85e275d703c3eec8cce6df}
    \item {\bf Control模块中的异常处理入口设置}:\url{http://47.94.142.165:8088/gitlab/PRJ11_NonExist/CPU/commit/35971df28621af7772590fdef9ffc4a80d20ba18}
    \item {\bf 增加ExtRAM}:\url{http://47.94.142.165:8088/gitlab/PRJ11_NonExist/CPU/commit/260859a2e6f35cb455533210daa7dc349d88a1a5}
    \item {\bf 串口接收端的一个bug}:\url{http://47.94.142.165:8088/gitlab/PRJ11_NonExist/CPU/commit/200a69e2bebbc24fb5d47f18f098c29b096652b9}
\end{enumerate}
