\chapter{硬件调试}

\section{简介概述}

只有做了充足的仿真,确保仿真结果运行完全正确,笔者才推荐进入这一阶段,硬件调试。

顾名思义,硬件调试,就是在硬件上的调试。

先将数据信息烧入到Flash和RAM中,利用Vivado编译生成出的bitstream文件烧入FPGA,注意这一步顺序不能相反,否则需要重新烧入FPGA。

相比仿真调试,硬件调试的过程非常痛苦,痛苦之处有以下三点:

(1)编译慢。无论多么细微的改动,完整的编译需要合成、实现、生成比特流三步,编译的速度和CPU核心数没有关系,只与单核主频有关,即使是高性能笔记本电脑,编译一次也需要整整10分钟。

(2)不灵活。如果接Thinpad板子上的clk时钟,clk时钟没有方式能到一个固定时间停止;如果手按时钟,仅适用于指令数量极少的时期,如果指令数目很多,很可能按一天也按不到想要的结果,更何况如果一不小心到了结果多按一次,唯一的解决办法只有按下rst,重新来过。

(3)信息不透明。想看到某一个数据信息,唯一的方法便是通过Thinpad开发板上16个LED灯和七段数码管显示出来,如果你想看另一个数据信息,请重新编译,重新来过。

但是即使硬件调试有这么多缺点,硬件调试也是不可缺少的必经过程,因为功能仿真无法解决时序问题,硬件运行正确才是我们最终的目标。

\section{具体实现}

接上板子降频后的clk时钟后(由于硬件延迟时间所限,一般CPU无法运行在板子自带时钟50MHz),由于做了充足的仿真测试,你的硬件测试结果应该和仿真结果一致。

如果不一致,推荐读者先对clk时钟降频再次测试,例如降频至1MHz,一般错误原因均为时序原因,如果降频后结果正确,说明是硬件条件限制导致主频不能过高,你只能选择优化CPU的代码。

如果降频仍然错误。那么只能进行硬件调试了。

硬件调试的方法是:

(1)CPU输入时钟改为clk按键

(2)将你想看到的调试信息,例如当前指令地址、当前指令内容等,通过MMU地址映射输出到LED灯和七段数码管中。

(3)精简数据,由于手按时钟不可能按太多次,尽可能保留精简的指令,例如功能测例只放入一组测例的一小部分等,精简后将数据拷入RAM和Flash

(4)编译并烧入FPGA

接下来,手按时钟,根据LED信息和七段数码管信息分析错误原因。