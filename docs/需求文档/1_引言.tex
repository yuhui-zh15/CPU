\chapter{文档说明}

\section{背景}

本项目的顶级需求为在Thinpad开发板上运行ucore操作系统。其衍生需求为设计一个基于MIPS 32架构的CPU,以实现ucore所需的46条指令、精确异常与外设调度。

本项目是计算机组成原理和软件工程课程的联合实验,项目需求方为计算机组成原理与软件工程课程。

本项目的承担方为NonExist小组,包括计55班张钰晖、计55班杨一滨、计54班周正平3位成员。

\section{编写目的}

本需求文档的编写目的如下:

\begin{enumerate}
    \item {\bf 需求分析:} 明确软件(ucore)对硬件(CPU)的具体需求,及需完成的功能
    \item {\bf 系统概述:} 对系统总体框架进行清晰完整的描述
    \item {\bf 学习目标:} 明确开发所需技术,订立学习目标
\end{enumerate}

其目标读者包括NonExist小组的开发者、成品CPU的使用者、项目的评估者。

\section{需求概览}

首先给出本项目的需求分析图:

\image[4in]{outline}{需求总览}

可以看出,本项目的需求可以作如下分层:

\begin{enumerate}
    \item {\bf ucore}:操作系统,编译后由大量MIPS指令组成。需要硬件实现MIPS指令系统的一个子集。
    \item {\bf 指令系统}:是MIPS指令系统的一个子集,需要硬件提供CPU(核心调度)与外部设备(组织集成)。
    \item {\bf CPU与外部设备}:是硬件最底层的实现,由大量基础模块组成。各个基础模块由开发板提供的FPGA或外围芯片实现。
\end{enumerate}

本文档也因而遵从自顶向下的分析流程,其组织结构与需求分析的层次相似。

\section{定义}

以下列出此项目涉及的部分英文简称及其含义(详细解释会在后文给出):

\tablethreeL{英文简称}{英文全称}{含义}
    MIPS & Microprocessor without  & 一种典型RISC指令集,本实验CPU的基本架构 \\
         & interlocked piped stages & \\
    CPU & Central Processing Unit & 中央处理器,负责硬件的核心调度 \\
    \midrule
    ALU & Arithmetic Logic Unit & 算术逻辑单元 \\
    HI/LO & High/Low Register & 寄存器,用于存储乘法运算结果(64位)的高、低各32位 \\
    \midrule
    MMU & Memory Management Unit & 内存管理单元,用于地址映射 \\
    TLB & Translation Lookaside Buffer & 快表,用于通过缓存加速MMU地址映射查表速度 \\
    VA & Virtual Address & 虚拟地址 \\
    PA & Physical Address & 物理地址 \\
    \midrule
    ROM & Read-Only Memory & 只读存储器,用于存储BootLoader \\
    RAM & Random Access Memory & 随机存储器,用于实现内存 \\
    Flash & Flash Memory & 快闪存储器,用于实现硬盘 \\
    VGA & Video Graphics Array & 视频图形阵列,一种显示器接口标准 \\
    HDMI & High Definition Multimedia Interface & 高清晰度多媒体接口,一种显示器接口标准 \\
\tableend
