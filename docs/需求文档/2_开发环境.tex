\chapter{开发环境}

这一部分对本项目涉及的开发环境进行说明。

本章分为如下几个部分:

\begin{enumerate}
    \item {\bf 硬件端开发环境}:对实验开发板及其配套EDA工具进行说明
    \item {\bf 软件端开发环境}:对ucore操作系统所需的编译流程、仿真方法、运行平台进行说明
\end{enumerate}

% 1. EDA工具
%
% 硬件端 平台  Thinpad开发板
% 软件端 平台  Windows
%       工具  开发工具:Vivado
%             串口工具:串口调试精灵
%             调试工具:二进制显示软件
%
% 2. MIPS运行环境
%
% 平台 Ubuntu
% 工具     编译工具:mips-sde-elf
%         仿真工具:qemu-thumips

\section{硬件端开发环境}

硬件端开发环境,主要包括实验开发板及配套的EDA开发工具。以下进行总结:

\begin{enumerate}
    \item {\bf Thinpad开发板(Late 2017版)}

    \tabletwoL{元件}{型号}
        FPGA & Xilinx Artix-7 xc7a100tfgg676-2L \\
        RAM & IS61WV102416ALL 32位 2块 各4MB \\
        FLASH & JS28F640J3D75 8MB \\
        HDMI & TFP410 \\
    \tableend

    \item {\bf EDA开发工具}

    \tabletwoL{软件}{运行平台}
        Vivado 2017.3 & Windows 10 64bit \\
        串口调试精灵 & Windows 10 64bit \\
        二进制显示软件 & Windows 10 64bit \\
    \tableend

    \item {\bf 云端硬件调试平台ThinpadCloud}

    通过该在线平台,可以远程对开发板进行操控,而无需随身携带,极大地方便了前期调试。其网址如下:

    \tabletwoL{平台}{网址}
        ThinpadCloud & http://thinpad.dynv6.net:8000 \\
    \tableend

\end{enumerate}

\section{软件端开发环境}

软件端开发环境,主要指ucore操作系统所需的编译、仿真、运行环境。以下进行总结:

\begin{enumerate}
    \item {\bf 编译}

    \tablefourL{工具}{版本号}{平台}{主要功能}
        mips-sde-elf-gcc & 4.6.3 & Ubuntu 16.04 x64 & 编译ucore \\
    \tableend

    编译应按照如下流程进行:

    \begin{enumerate}
        \item {\bf 下载ucore-thumips源代码}

        这里我们需下载ucore-thumips,是ucore操作系统的MIPS移植版本。
        从其GitHub仓库\footnote{https://github.com/chyh1990/ucore-thumips}下载即可,不再赘述。

        \item {\bf 安装mips-sde-elf-gcc}

        \begin{enumerate}
            \item 从Sourcery CodeBench官网\footnote{https://sourcery.mentor.com/GNUToolchain/release2189}下载文件mips-\{版本号\}-mips-sde-elf.bin
            \item 执行\texttt{./mips-\{版本号\}-mips-sde-elf.bin}以执行默认安装向导,这一步骤之后应完成环境变量的配置。可输入\texttt{mips-sde-elf-gcc --version}进行验证。
            \item 如环境变量配置不成功,应手动在$\sim$/.bashrc中加入\path{export PATH="{path/to/mipsgcc}/bin:$PATH"}并重启终端。
        \end{enumerate}

        \item {\bf 修改Makefile}

        \tablethreeL{需修改的变量}{修改后的值}{说明}
            GCCPREFIX & GCCPREFIX:=mips-sde-elf- & 用于指定编译工具 \\
            \midrule
            ON\_FPGA & y/n & y表示为FPGA编译,用于硬件上实际运行\\
                    & & n表示为qemu模拟器编译,用于仿真调试 \\
            \midrule
            USER\_APPLIST & sh, ls, cat, ... & 指定ucore可支持的应用程序列表 \\
                    & & 当ON\_FPGA=y时,默认仅支持sh命令, \\
                    & & 后期调试通过后可增加其他命令 \\
        \tableend

        \item {\bf make}

        输入\texttt{make}命令以编译。

    \end{enumerate}

    \item {\bf 仿真}

    \tablethreeL{工具}{平台}{主要功能}
        qemu-thumips & Ubuntu 16.04 x64 & 1. 通过修改thumips\_insn.txt,确定指令子集能否支持ucore \\
                    & &                    2. 对比观察ucore在开发板和在模拟器上的运行输出,推断错误原因 \\
    \tableend

    仿真应按如下流程进行:

    \begin{enumerate}
        \item {\bf 下载qemu-thumips源代码}

        这里我们需下载qemu-thumips,是qemu模拟器的MIPS移植版本。
        从其GitHub仓库\footnote{https://github.com/chyh1990/QEMU-thumips}下载即可,不再赘述。

        \item {\bf 配置}

        输入\texttt{./configure –target-list=mipsel-softmmu}命令即可。

        \item {\bf make}

        输入\texttt{make}命令以编译。

        需注意,这一步依具体平台可能出现链接错误。笔者出现的链接错误位于qemu-timer.o,最终通过修改Makefile与Makefile.target解决:

        \begin{lstlisting}
        # Makefile, line 37
        LIBS+=-lz -lrt -lm $(LIBS_TOOLS)

        # Makefile.target, line 37
        LIBS+=-lz -lrt -lm
        \end{lstlisting}

    \end{enumerate}

    \item {\bf 运行}

    ucore经过编译后,应将生成的obj/ucore-kernel-initrd(注意根目录下flash.img是一个软链接文件,链接指向ucore-kernel-initrd)文件写入FLASH中。详见``ucore需求分析''的``Boot阶段''一节。

\end{enumerate}
