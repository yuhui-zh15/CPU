\chapter{ucore需求分析}

正如引言部分所述,``运行ucore操作系统''为本项目的顶级需求。这一部分将主要介绍

\section{Boot}

\section{运算跳转}

\section{地址映射}

\image[3in]{mips_mmu.PNG}{MIPS标准地址映射}

地址映射在硬件上由MMU(Memory Manage Unit)完成,其最重要的意义体现在2方面:

\begin{enumerate}
    \item {\bf 内存管理(面向软件)}:使进程在寻址时可以超过物理内存大小,且多个进程的地址空间互不干扰
    \item {\bf 外设调度(面向硬件)}:面向CPU统一各个外设的访问接口,使之均可使用访存指令进行读写
\end{enumerate}

这一部分将深入解析ucore的地址映射标准,从而明确对硬件MMU(Memory Manage Unit)的需求。

\subsection{相关文件}

首先列出ucore定义地址映射的关键文件:

\tablefourL{文件}{关键变量}{值}{简介}
    boot/bootasm.S & FLASH\_START & 0xBE000000 & FLASH的起始虚拟地址 \\
                   & FLASH\_SIZE &  0x01000000 & FLASH的地址大小 \\
    \midrule
    kern/mm/memlayout.h & KMEMSIZE & 1M & 内存总大小 \\
    kern/include/mips\_vm.h &  MIPS\_KUSEG & 0x00000000 & kuseg段起始地址 \\
                            &  MIPS\_KSEG0 & 0x80000000 & kseg0段起始地址 \\
                            &  MIPS\_KSEG1 & 0xa0000000 & kseg1段起始地址 \\
                            &  MIPS\_KSEG2 & 0xc0000000 & kseg2段起始地址 \\
    \midrule
    include/thumips.h & COM1 & 0xBFD003F8 & 串口虚拟地址 \\
\tableend

\subsection{内存管理:TLB}

内存管理需求主要面向运行在操统之上的应用软件。首先,应用程序的大小不应受到物理内存的限制,例如开发板上的RAM总大小仅有8M,但对于32位机器而言,可用虚拟地址空间可达$2^{32} = 4G$;其次,各应用程序的地址空间应各自独立。

\paragraph{页表}
操统因而需维护页表(Page Table),以实现从虚拟地址到物理地址的转换。具体而言,每当程序访问内存时,需进行2次访问:
\begin{enumerate}
    \item {\bf 查询页表}:读取页表,查找对应的物理地址
    \item {\bf 获取数据}:访问该物理地址,获取数据
\end{enumerate}

\image[5in]{tlb_entry}{TLB表项}

\paragraph{TLB}
由于内存访问具有局部性,如能将页表最近被访问的一部分以CPU内部的逻辑单元存储,则访存效率可获得极大提高。TLB(Translation Lookaside Buffer)就是这样一种高速缓存。具体而言,在加入TLB之后,程序访存流程如下:

\begin{enumerate}
    \item {\bf 查询TLB}:根据虚拟地址的高20位(32减去页位数12),查找TLB中的表项
    \item {\bf TLB HIT}: 如TLB中有此表项,则直接其对应的物理地址
    \item {\bf TLB MISS}:如TLB中无此表项,则再进一步访问内存查询页表,获取物理地址之后,将其作为一个新表项写入TLB
\end{enumerate}

\subsection{外设调度}

在CPU核心之外,运转着ROM、RAM(相当于内存)、FLASH(相当于硬盘)、串口(相当于标准输出)、VGA(相当于显示屏)等多个外部设备。它们的用途、接口、访问时序各不相同。

然而,对于操作系统与CPU而言,它们的访问接口是统一的:均使用L/S型指令实现收/发数据。这个过程中需要进行地址映射。

\paragraph{RAM}
除操统文件中的定义、``内存管理''小节的说明外,考虑到开发板上提供了2块大小为4M的RAM(Base RAM与Ext RAM),可将KMEMSIZE改为8M,从而提供更大的内存空间。

\paragraph{FLASH、串口}
见操统文件中的定义。

\paragraph{ROM}
在Boot阶段,CPU复位时PC位于VA 0xBFC00000处,因而该虚拟地址应映射到ROM。此外,考虑BootLoader指令较少,为ROM分配1KB的地址空间即可。

\paragraph{VGA}
此外,项目计划实现拓展功能VGA,其屏幕大小为800x600,故而需要480000 B的地址空间用于显存。姑且在kseg1段中分配一段开始于VA 0xBA000000、大小为512KB的地址空间留作显存。

\subsection{需求分析}

综上所述,MMU需要将VA的kuseg、kseg2段通过TLB映射到内存(RAM),在kseg0段通过抹去最高位直接得到RAM中的PA,在kseg1段将部分地址映射到除RAM外的各个外设。

以下给出最终地址映射方案:

\tablefourL{段(权限)}{虚拟地址}{映射目标(物理地址)}{备注}
    kuseg(用户态) & 0x00000000 - 0x7FFFFFFF & RAM(通过查询TLB动态确定) & 用户程序 \\
    \midrule
    kseg0(内核态) & 0x80000000 - 0x807FFFFF & RAM(0x00000000 - 0x007FFFFF) & 开机时存放操作系统 \\
    \midrule
    kseg1(内核态) & 0xBE000000 - 0xBEFFFFFF & FLASH & 关机时存放操作系统 \\
                 & 0xBFC00000 - 0xBFC00FFF & ROM & 存放BootLoader \\
                 & 0xBFD003F8 - 0xBFD003FC & 串口 & 串口数据/串口状态 \\
                 & 0xBA000000 - 0xBA080000 & VGA & 显存,用于显示图像 \\
    \midrule
    kseg2(内核态) & 0xC0000000 - 0xFFFFFFF & RAM(通过查询TLB动态确定) & \\
\tableend

\section{异常处理}

在MIPS32架构中,有一些事件会打断程序的正常执行流程。一些由{\bf 外部事件}触发,如串口有数据待读入,称为\emph{中断};另一些则由CPU{\bf 内部指令}引起,如算术溢出、系统调用等。这些事件统称为\emph{异常}。

以下对异常处理流程进行归纳,并总结ucore涉及的所有异常,从而明确对硬件的需求。

\subsection{相关文件}

以下列出ucore异常处理的部分关键文件:

\tablethreeL{文件}{关键函数}{简介}
    trap/vector.S & \_\_exception\_vector & 异常处理向量,汇集各入口 \\
    trap/exception.S & ramExcHandle\_tlbmiss & TLB异常处理入口 \\
                     & ramExcHandle\_general & 通用异常处理入口 \\
                     & common\_exception & 保存现场、调用操统异常处理代码、恢复现场 \\
    trap/trap.c & mips\_trap & 操统异常处理代码 \\
                & trap\_dispatch & 被mips\_trap调用,分类处理各种异常 \\
\tableend

\subsection{异常处理流程}

% <TODO>: Please check
% <TODO>: 泳道图?

\paragraph{硬件端(CPU)}

\begin{enumerate}
    \item {\bf 异常响应}:硬件检测异常,并将异常原因、类型等存入CP0相关寄存器中
    \item {\bf 异常处理}:将PC跳转到CP0 Ebase寄存器所指示的操统异常处理入口地址,并禁用异常检测
    \item {\bf 异常返回}:执行ERET指令,跳转回被异常打断的指令,重新使能异常检测
\end{enumerate}

\paragraph{软件端(ucore)}

\begin{enumerate}
    \item {\bf 异常响应}:保存现场(通用寄存器等)至内存中
    \item {\bf 异常处理}:从异常处理入口处开始,根据异常类型,执行异常处理代码
    \item {\bf 异常返回}:从内存中恢复现场,并使用ERET指令返回
\end{enumerate}

\subsection{需求分析}

根据上述分析,只需针对ucore能够处理的那些异常,在硬件上加以实现,即可满足异常处理需求。

阅读trap.c : trap\_dispatch函数可知ucore处理的所有异常类型。它们通过CP0 Cause寄存器的ExcCode(异常号)字段加以区分:

\paragraph{中断}

中断由CP0 Cause寄存器的IP标志位(中断号)进一步区分,包括如下2种:
\tablefourL{异常名(异常号)}{中断名(中断号)}{硬件触发条件}{ucore处理流程}
    Interrupt(0) & 时钟中断(7) & CP0 Compare与Count寄存器的值相等 & 进程调度后重启时钟 \\
                   & 串口中断(4) & 串行接口处有数据待写入 & 读串口并写入stdin \\
\tableend

\paragraph{TLB MISS}

在访存load或store时TLB中无匹配表项时产生,包括如下2种异常:
\tablethreeL{异常名(异常号)}{硬件触发条件}{ucore处理流程}
    TLBL(2) & 访存load时TLB中无匹配表项 & 首先根据CP0 BadVAddr寄存器确定缺失地址,\\
                                    &&  然后通过设置CP0相关寄存器构造新的TLB表项,\\
                                    &&  最后使用tlbwr指令随机选择一个位置,重填该项 \\
    TLBS(3) & 访存store时TLB中无匹配表项 & 同上 \\
\tableend

\paragraph{系统调用}

系统调用由Syscall序号进一步区分,其类型在操作系统中定义,硬件无需关心其具体实现:
\tablethreeL{异常名(异常号)}{硬件触发条件}{ucore处理流程}
    Syscall(8) & 执行指令SYSCALL & 根据Syscall序号,调用syscall/syscall.c中相应的处理代码 \\
\tableend

\paragraph{其他异常}

此类异常无需在硬件上实现。因为它们在ucore中的实现均为简单报错,如发生在用户态,则用户进程退出;如发生在内核态,则系统panic。总而言之,即使硬件实现了这些异常,亦无法使ucore在这些情况下正常运行。为完整起见,亦在此列出:

\tablethreeL{异常名(异常号)}{硬件触发条件}{ucore处理流程}
    ADEL(4) & 访存load时地址未对齐 & 简单报错 \\
    ADES(5) & 访存store时地址未对齐 & 简单报错 \\
    RI(10) & 无效指令 & 简单报错 \\
    CPU(11) & 协处理器不可用 & 简单报错 \\
    OV(12) & 算术溢出 & 简单报错 \\
\tableend

% \subsection{新增指令需求}
%
% 除了此前分析的计算指令、跳转指令、访存指令外,异常部分新增指令如下:
%
% \tablethreeL{指令名称}{指令类型}{说明}
%     MFC0 & 特权指令 & 读CP0:维护CP0相关寄存器的值,以便异常处理使用 \\
%     MFC0 & 特权指令 & 写CP0:维护CP0相关寄存器的值,以便异常处理使用 \\
%     SYSCALL & 陷入指令 & 触发系统调用异常,调用操统处理代码 \\
%     ERET & 特权指令 & 异常处理完毕时返回,并重启异常检测使能 \\
%     TLBWR & 特权指令 & 随机选择一个位置,重填缺失的TLB表项 \\
%     TLBWI & 特权指令 & <TODO>: 用到了吗? \\
% \tableend
