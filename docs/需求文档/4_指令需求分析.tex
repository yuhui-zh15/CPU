\chapter{指令系统需求分析}

指令系统是硬件(CPU)对ucore唯一透明可见的接口,也因而成为在``运行ucore''之下的第二级需求。

在对ucore进行需求分析后可知,在指令系统这一层面,一共有45条需要实现的指令。

本章包括如下几个部分:

\begin{enumerate}
    \item {\bf 概述}:对五级流水线架构及其运行方式作一简要概述
    \item {\bf 算术逻辑指令}:共22条,包括加减乘、与或非、移位等指令
    \item {\bf 分支跳转指令}:共10条,包括分支(B)与跳转(J)指令
    \item {\bf 访存指令}:共5条,包括读取(L)与写入(S)指令
    \item {\bf 移动指令}:共2条,包括向HILO寄存器的移动
    \item {\bf 陷入指令}:共1条,包括SYSCALL
    \item {\bf 特权指令}:共5条,包括对CP0的访问、异常返回及TLB异常时使用的指令
    \item {\bf 总结}:以上指令对硬件需求的扼要总结
\end{enumerate}

本章除概述与总结外,每节遵从以下介绍流程:

\begin{enumerate}
    \item {\bf 功能}:简要描述此类指令的功能
    \item {\bf 硬件需求}:为实现此类指令,对硬件的结构和功能需求
    \item {\bf 异常}:此类指令可能触发的异常
\end{enumerate}

\section{概述}

\subsection{五级流水线}

本次项目计划实现基于\emph{五级流水线}的指令系统。何为五级流水线?具体来说,体现在以下2个方面:

\begin{enumerate}
    \item {\bf 每条指令被拆分为5个步骤}:共包括5个硬件单元,每个单元负责其中的一个环节。
    \tablethreeL{名称}{代号}{简介}
        取指 & IF & 从指令存储器中读取指令 \\
        译码 & ID & 指令译码,同时读取寄存器 \\
        执行 & EX & 执行操作,或计算地址 \\
        访存 & MEM & 进行访存操作 \\
        回写 & WB & 将计算结果写入寄存器 \\
    \tableend

    \item {\bf 5个硬件单元并行执行}:在任何一个时钟周期内,上述的5个硬件单元分别在处理第$n+5, n+4, n+3, n+2, n+1$条指令的第1, 2, 3, 4, 5个阶段(如下图所示):
    \image[4in]{pipeline}{流水线结构}

    硬件单元的并行执行同时有其优点和缺点:

    \begin{itemize}
        % TODO:Please check
        \item {\bf 优点}:\emph{并行结构加速指令的执行}。设每条指令每个步骤的平均执行时间为$\triangle t$,则对于一个多周期CPU(无流水线结构)而言,$n$条指令所需时间为$5n\triangle t$;
        而对于流水线结构而言,$n$条指令仅需$(n+4)\triangle t$。可以看出,当指令数目很大时,流水线结构可带来约4倍左右的性能提升。

        \item {\bf 缺点}:\emph{并行结构同样带来了冒险问题}。具体地,有2种情形由此产生:
            \begin{itemize}
                \item {\bf 数据冒险}:例如后一条指令的ID阶段需读取前一条指令在WB阶段才写入的寄存器值,但这在时序上矛盾。
                \item {\bf 控制冒险}:例如跳转指令在ID阶段才能确定目标地址,然而此时其下一条指令已进入流水线的IF阶段。
            \end{itemize}
    \end{itemize}

    % TODO:冒险放在设计文档怎么样?需求文档要不就不展开了

\end{enumerate}

\subsection{需求总述}

总体来看,指令系统对硬件的需求如下:

\begin{enumerate}
    \item {\bf 流水线架构}:实现流水线框架,使得硬件在每个周期可以并行执行5条指令。
    \item {\bf 指令实现}:在流水线框架上实现ucore所需的45条指令。
    \item {\bf 运行控制}:能够对流水线进行暂停、插空周期或清空等控制干预。
    \item {\bf 解决冒险}:解决数据冒险、控制冒险。
\end{enumerate}

\section{算术逻辑指令}

\subsection{功能}

算术逻辑指令共22条,包括加减乘、与或非、移位等,总结如下:

\tablethreeL{指令}{运算}{描述}
    ADDIU、ADDU & A + B & 无符号数加法 \\
    SUBU & A - B & 无符号数减法 \\
    MULT & A $\times$ B & 乘法 \\
    SLT、SLTI、SLTIU、SLTU & A < B & 符号数比较、无符号数比较 \\
    \midrule
    AND、ANDI & A and B  & 与 \\
    OR、ORI & A or B & 或 \\
    NOR & A nor B & 或非 \\
    XOR、XORI & A xor B & 异或 \\
    SLL、SLLV & A sll B & 逻辑左移 \\
    SRL、SRLV & A srl B & 逻辑右移 \\
    SRA、SRAV & A sra B & 算术右移 \\
    \midrule
    LUI & A = imm || $0^{16}$ & 加载立即数至寄存器 \\
\tableend

\subsection{硬件需求}

这部分对硬件的需求主要体现在ALU上,也即流水线的第3部分(EX)。具体地,ALU需接受2个32位整数及相关控制信号作为输入,并以1个32位整数作为输出。控制信号决定了ALU执行的运算;
此外,运算结果应在WB阶段被写入寄存器。

特别地,乘法运算的结果是一个64位整数,因此需要HI/LO寄存器分别用于存储乘积的高、低32位,而不能存入通用寄存器。这也衍生出了相应的移动指令MFHI等。

重点总结如下:

\tabletwoL{硬件单元}{用途}
    ALU & 位于EX阶段,用于计算2个32位整数的运算结果 \\
    HI/LO寄存器 & 用于存储乘积 \\
\tableend

\subsection{异常}

% TODO:有溢出异常么?
无。

\section{分支跳转指令}
<总表>
    \subsection{功能}
    \subsection{硬件需求}
    \subsection{异常}

\section{访存指令}
<总表>
    \subsection{功能}
    \subsection{硬件需求}
    \subsection{异常}

\section{陷入指令}
<总表>
    \subsection{功能}
    \subsection{硬件需求}
    \subsection{异常}

\section{特权指令}
<总表>
    \subsection{功能}
    \subsection{硬件需求}
    \subsection{异常}

\subsection{空指令}
<总表>
