\chapter{外设需求分析}

在上一章中,将本次工程需要实现的硬件系统划分为CPU、外设2大部分,并对其中的CPU部分进行了需求分析。本章分析外设部分。

本章分为如下几个部分:

\begin{enumerate}
    \item {\bf ROM}:只读存储器,用于存储BootLoader
    \item {\bf RAM}:随机存取存储器,用作内存
    \item {\bf FLASH}:闪存,用作硬盘
    \item {\bf 串口}:串行接口,用作通信(标准输入输出)
    \item {\bf VGA}:光栅扫描显示器,用作图像显示
    \item {\bf 总结}:对各个外设的需求进行总结
\end{enumerate}

除总结外,每部分主要介绍其参数、文档信息及功能。

\section{ROM}

ROM为只读存储器,需提供读支持。

ROM需利用FPGA上的逻辑单元实现,在Thinpad开发板上不额外提供元件。其具体需求在``ucore需求分析''的``Boot阶段''一节已经详述,此处不再赘言。

\section{RAM}

RAM在系统中充当内存,需提供读写支持。

Thinpad开发板上提供的RAM包括BaseRAM(基础内存)与ExtRAM(扩展内存)2部分,其共同特点是断电后数据会消失。其大小均为4MB,每次可读或写32位。

在寻址时,BaseRAM与ExtRAM均通过MMU统一进行地址映射。

在硬件端,我们需要将RAM作为一个黑盒子集成至CPU。这要求阅读其文档(``使用说明书''),明确这个黑盒子提供的读、写、使能信号等接口。这样,在CPU的MMU部分便可实现接口的衔接:

\tablethreeL{接口信号}{含义}{接口类型}
    ram\_data[31:0] & data,数据总线,位宽32位,通过选择信号选择读哪些byte & inout \\
    ram\_addr[19:0]	& address,地址总线,位宽20位,故RAM总大小为$32 \times 2^{20} = 4MB$ & output \\
    ram\_be\_n[3:0]	& select,选择信号,位宽4位,选择32bit=4byte的哪些位进行读取或写入	& output \\
    ram\_ce\_n	& chip enable,第一次下降有效 & output \\
    ram\_oe\_n	& output enable,低电平为读入,置0表示使能 &	output \\
    ram\_we\_n	& write enable,低电平为写入,置0表示使能 & <TODO> \\
\tableend

文档中提供的读写时序图如下,CPU在集成时需注意协调满足:

\image[5in]{RAM_r}{RAM读时序}
\image[5in]{RAM_w}{RAM写时序}

\section{Flash}

FLASH在系统中充当硬盘,用于存储ucore,仅需提供读支持。

Thinpad开发板上提供的FLASH在掉电后不会丢失数据,其大小为8MB,采用字编址,每个地址空间存8bit,每次可选择读16位或8位。

在寻址时,FLASH通过MMU统一进行地址映射。

在硬件端,我们需要将FLASH作为一个黑盒子集成至MMU。这同样要求阅读其文档:

\tablethreeL{接口信号}{含义}{接口类型}
    flash\_a[22:0] & address,地址总线,位宽23位,故FLASH总大小为$8 \times 2^{23} = 8MB$ & output \\
    flash\_rp\_n & reset,高电平为正常操作,低电平节电模式 & output \\
    flash\_oe\_n & output enable,低电平为读入,置0表示使能 & output \\
    flash\_data[15:0] & data,数据总线,位宽16位,一次读16bit & inout \\
    flash\_ce\_n & chip enable,第一次下降有效 & output \\
    flash\_byte\_n & byte enable,高电平一次读16bit,低电平一次读8bit & output \\
    flash\_we\_n & write enable,低电平为写入,置0表示使能 & output \\
    flash\_vpen & 写保护,置0表示使能 & output \\
\tableend

文档中提供的读时序图如下,CPU在集成时需注意协调满足:

\image[6in]{FLASH_r}{FLASH读时序}

\section{串口}

串行接口在系统中负责通信,需提供读写支持。ucore读、写标准输入均通过此处。

Thinpad开发板上提供的串口通过UART芯片进行访问,需要<TODO>位用作数据发送接收端口,及另<TODO>位用于串口状态标记。

寻址时,串口通过MMU统一进行地址映射。

在硬件端,我们需要将串口作为一个黑盒子集成至MMU。不同的是,串口的控制代码(async.v)已经由助教提供,并封装为2个模块。其接口如下所示:

\tablethreeL{接口信号}{含义}{接口类型}
    clk & 时钟 & input \\
    TxD\_start & 开始标志,开始传输数据前需有1周期保持为高电平 & input \\
    TxD\_data[7:0] & 发送至FPGA外部的数据,每次可发送8位 & input \\
    TxD & 串行化后,发送至FPGA外部的数据,每次发送1位 & output \\
    TxD\_busy & <TODO> & output \\
\tablecapend{模块1:async\_transmitter接口}

\tablethreeL{接口信号}{含义}{接口类型}
    clk & 时钟 & input \\
    RxD & 串行化的,来自FPGA外部的数据,每次发送1位 & input \\
    RxD\_data\_ready & 串口是否有准备好的数据待接收 & output \\
    RxD\_data[7:0] & 去串行化后,来自FPGA外部的数据,每次可接收8位 & output \\
    % TODO:Please Check
    RxD\_idle & 当一段时间内无接收数据时置1 & output \\
    RxD\_endofpacket & 当一个数据包被识别时置1一个时钟周期 & output \\
\tablecapend{模块2:async\_receiver接口}

网站http://www.fpga4fun.com/SerialInterface.html中提供的读写概念图如下:

\begin{figure}[H]
    \centering
    \miniimage[3in]{async_trans}{模块1:async\_transmitter}
    \miniimage[3in]{async_recv}{模块2:async\_receiver}
\end{figure}

当串口传输数据时,需将其串行化后进行发送,遵循如下步骤:

\begin{enumerate}
    \item 无数据时,串口发送idle信号(=1)。
    \item 每发送1byte = 8bit前,串口发送start信号(=0)。
    \item 依次发送该字节的8位。
    \item 每发送1byte = 8bit后,串口发送stop信号(=0)。
\end{enumerate}

例如,发送字节0x55时,串口发送数据的时序图如下:

\image[6in]{serial}{串口发送字节0x55}

\section{VGA}

VGA为图像显示设备,需要显存提供读写支持。该设备的具体需求为通过逐行扫描的方式实现图像显示。

Thinpad开发板提供了HDMI接口,使用时可与自己的HDMI显示屏进行连接。本次实验所使用的显示屏分辨率为$600 \times 800$像素,屏幕刷新率为<TODO>Hz。
在将图像像素编码为8位(其中R/G/B各3/3/2位)后,输出至该引脚,即可实现图像显示功能。

寻址时,显存通过MMU统一进行地址映射。

在硬件端,我们需要将VGA集成至MMU。其中逐行扫描屏幕的代码(vga.v)已经由助教提供,我们只需再实现其中的显存模块。vga.v的接口如下所示:

\tablethreeL{接口信号}{含义}{接口类型}
    clk & 时钟 & input \\
    hsync & 水平同步信号 & output \\
    vsync & 竖直同步信号 & output \\
    hdata & 水平像素位置 & output \\
    vdata & 竖直像素位置 & output \\
    data\_enable & 数据使能信号,使能时置1 & output \\
\tablecapend{vga.v模块接口}

\section{总结}

以下对各个外设的需求进行总结:

\tablethreeL{外设}{功能需求}{读写支持}
    ROM & 只读存储器,能用于烧录存储BootLoader & R \\
    RAM & 随机访问存储器,相当于内存,能用于实现运行时数据存储  & R/W \\
    FLASH & 闪存,相当于硬盘,能用于存储ucore & R \\
    串口 & 串行并行转换接口,能实现收发数据通信 & R/W \\
    VGA & 图像显示接口,能逐行扫描显示图像 & R/W \\
\tableend
